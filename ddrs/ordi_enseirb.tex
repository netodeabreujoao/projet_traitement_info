\section{Utilisation de DELL T1700/3620 à l'école}

Dans cette section, nous évaluons le bilan carbone lié à l'utilisation de 600 ordinateurs DELL T1700/3620 à l'ENSEIRB-MATMECA sur une période de 5 ans, correspondant à leur cycle de vie.

Pour ce faire, nous avons modélisé dans \textit{openLCA} les 600 ordinateurs, en utilisant les \textit{flows} créés précédemment. Nous avons également pris en compte l’électricité consommée pour leur fonctionnement.

D'après nos recherches, chaque ordinateur consomme en moyenne 290~W. Sur 5 ans, cela correspond à une consommation énergétique totale de :

\[
E_{\text{totale}} = 600 \times 290~\text{W} \times 24~\text{h/j} \times 365~\text{j/an} \times 5~\text{ans} = 8{,}17 \times 10^8~\text{Wh}
\]

Ce qui nous donne, selon les résultats obtenus dans \textit{openLCA}, un bilan carbone estimé à :

\[
\text{Bilan carbone total} \approx 2{,}22 \times 10^5~\text{kgCO}_2~\text{eq}
\]

\subsection{Analyse des phases du cycle de vie}

Le cycle de vie d’un ordinateur se divise principalement en deux grandes phases :

\begin{itemize}
    \item \textbf{Phase de fabrication} : extraction des matières premières, fabrication des composants, assemblage, et transport.
    \item \textbf{Phase d’usage} : consommation énergétique, maintenance, et mise à niveau des logiciels.
\end{itemize}

Selon l’étude de DELL~\citep{dell2018carbon}, la phase de fabrication comprend les contributions des matériaux, de la fabrication proprement dite, ainsi que de la distribution. La phase d’usage, quant à elle, regroupe l'utilisation ainsi que la gestion de la fin de vie du matériel.

D’après cette étude et notre propre modélisation, les composants les plus impactants en termes d’émissions sont la carte mère et la carte graphique.

\subsection{Bilan carbone par étudiant}

Nous cherchons maintenant à estimer l’empreinte carbone associée à l’usage informatique d’un étudiant sur ses trois années de formation à l’ENSEIRB-MATMECA. Pour cela, nous procédons aux calculs suivants :

\[
\frac{2{,}22 \times 10^5~\text{kgCO}_2~\text{eq}}{600} = 370~\text{kgCO}_2~\text{eq par étudiant pour 5 ans}
\]

\[
\frac{3}{5} \times 370 = 222~\text{kgCO}_2~\text{eq sur 3 ans}
\]

Ainsi, l’impact carbone moyen associé à l’usage informatique d’un étudiant sur l’ensemble de sa scolarité est estimé à 174~kgCO\textsubscript{2}~eq.
