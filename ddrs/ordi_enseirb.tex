\section{Utilisation de DELL T1700/3620 à l'école}

Dans cette partie, nous allons calculer le bilan carbone lorsqu'on utilise 6OO ordinateurs DELL T1700/3620 à l'Enseirb-Matmeca durant 5 ans, c'est-à-dire leur cycle de vie.

Donc sur openLCA, nous avons mis 600 flows d'ordinateur DELL T1700/3620 que nous avons créer dans la partie précédente et l'éléctricité utilisé pour le fonctionnement des ordinateurs.

Après des recherches sur internet, nous avons trouvé que l'ordinateur consomme 290 Watts. Ce qui donne 8,17 $ \times 10^8$ Wh.

On obtient ainsi un bilan carbone de (à verifier ) 1,74 $\times 10^5$ kgCO2 eq.



Lors de la phase de fabrication, nous avons l'extraction des matières premières, la fabrication des composants, l'assemblage et le transport. Alors que pendant la phase d'usage, nous avons la consomation énergétique, la maintnance et la mise à niveaux des logicielles.

Or dans l'étude de DELL ( \citep{dell2018carbon} ), la phase de fabrication est composée des contributions des matériaux, de la fabrication et de la distribution et la phase d'usage est composée de l'utilisation et de la gestion de la fin de vie.


D'après l'étude de DELL ( \citep{dell2018carbon} ) et de notre bilan de carbonne, le plus impactant dans l'ordinateur est la carte mère et carte graphique.



Pour calculer le bilan carbone d'un étudiant pour la partie informatique durant ses 3 années d'études, nous devons faire les calculs suivants:

\begin{equation*}
    \frac{1,74 \times 10^5 \text{ kgCO2 eq}}{600} = 290 \text{kgCO2 eq par étudiant pendant 5 ans} 
\end{equation*}

\begin{equation*}
    \frac{3}{5} \times 290 = 174 \text{ kgCO2 eq sur 3 ans}
\end{equation*}