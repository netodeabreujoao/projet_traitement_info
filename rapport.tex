\documentclass[a4paper,12pt]{article}

\usepackage[utf8]{inputenc}
\usepackage[french]{babel}
\usepackage{footnote}
\usepackage{hyperref}
\usepackage{graphicx}
\usepackage{tikz}
\usepackage{fancyhdr}
\usepackage[numbers]{natbib}
\usepackage{amssymb}
\usepackage{amsmath}
\usepackage{listings}
\usepackage{xcolor}
\usepackage{float}
\usepackage[T1]{fontenc}
\usepackage{natbib}
\usepackage{subcaption}


\usetikzlibrary{positioning}
\usetikzlibrary{calc}
\usetikzlibrary{math}
\usetikzlibrary{arrows}

\pagestyle{fancy}

\textheight 26cm      
\textwidth 16.0cm      
\oddsidemargin 2.5cm            
\evensidemargin 2.5cm          
\addtolength{\oddsidemargin}{-2.5cm}
\addtolength{\evensidemargin}{-2.5cm}
\topmargin 0 cm      
\addtolength{\topmargin}{-2cm}  

\fancyhfoffset[L]{0.1in} % Ajuster la marge gauche de l'en-tête
\fancyhfoffset[R]{0.1in} % Ajuster la marge droite de l'en-tête

\renewcommand{\headrulewidth}{1pt}
\fancyhead[R]{Année universitaire: 2024-2025}
\fancyhead[L]{Pedro Henrique Rodriguez Russo - Jo\~ao Pedro Neto de Abreu}

\renewcommand{\footrulewidth}{1pt}
\fancyfoot[C]{\thepage} 
\fancyfoot[L]{ENSEIRB-MATMECA}
\fancyfoot[R]{TP Analyse du Cycle de Vie }

\newcommand\blankpage{%
    \null
    \thispagestyle{empty}%
    \addtocounter{page}{-1}%
    \newpage}



\title{Application K-Moyennes}

\author{NETO DE ABREU Jo\~ao Pedro \\ RODRIGUEZ RUSSO Pedro Henrique}
\date{}

\begin{document}

\maketitle

\begin{center}
  \large
  EISIS102 - Traitement de l'Information \\
  Département Informatique\\
  S5 - Année 2024/2025\\
  \vfill
  \includegraphics[width=0.2\textwidth]{enseirb-matmeca.png}
\end{center}

\newpage

\tableofcontents

\newpage

\section{Introduction}
Ce rapport concerne le traitement de données publiques issues d'un site de base données pour appliquer au machine learning \footnote{\url{https://archive.ics.uci.edu/}}. Pour analyser ces données, qui sont de types ratio, nous avons choisi d'utiliser deux méthodes :\\
\begin{enumerate}
\item On appliquera d'abord un ACP afin de réduire le nombre d'attribus pour ensuite appliquer l'algorithme des K-moyennes pour trouver les 3 cépages differents\\
\item On applique directement l'algorithme des K-moyennes
\end{enumerate}
\vspace{1cm}
Dans la section \ref{sec:presentation1}, nous présentons la base de données utilisée.Nous décrivons alors la méthode d'analyse de donnée considérée en section \ref{sec:presentation2}.Puis, on analysera les résultats obtenues dans la séction \ref{sec:analyse}. Une conclusion sur le travail réalisé est enfin proposée en séction \ref{sec:conclusion}

\section{Présentation des données}
\label{sec:presentation1}
Cette base de données résulte d'une étude chimique des vins produit dans la même région italienne. Cette base de données rassemble des données concernant des vins issus de trois variétés distinctes. Les informations ont été collectées afin d'analyser la composition chimique et de fournir un cadre pour des analyses statistiques, en particulier en apprentissage automatique. Les informations sont issues de l'étude de 13 composés chimiques présents dans chaque type de vin.

\section{Présentation de la méthode}
\label{sec:presentation2}
tkt

\section{Analyse des résultats obtenues}
\label{sec:analyse}
tkt

\section{Conclusion}
\label{sec:conclusion}
tkt

\section*{Références}
\label{sec:référence}

\end{document}
